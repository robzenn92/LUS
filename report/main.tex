\documentclass[a4paper,8pt,oneside]{article}

% \usepackage{minted}
% \usepackage{titlesec}
\usepackage{tikz}
\usepackage{pgfplots}
% \usepackage{url}

% \titleformat{\section}{\normalfont\fontsize{10}{10}\bfseries}{\thesection}{1em}{}

\author{Zen Roberto}
\title{\vspace{-4em} \LARGE First Project - Report}
\date{\today}

\begin{document}

% \maketitle

% GOOD Report that includes: Data Analysis
% Evaluation
% Comparison of different:
% Feature Sets
% Tool Configurations

\begin{titlepage}
	\pagestyle{empty}

	\begin{center}
		{\bfseries\Large {\huge U}NIVERSITY OF {\huge T}RENTO}

		\vspace{0.2cm}

		{\large Department of Information Engineering and Computer Science}

		\vspace{0.5cm}

		\begin{center}
			\includegraphics[width=0.3\textwidth]{res/unitn}
		\end{center}

		\vspace{2.5cm}

		{\Large \bfseries {{\huge L}ANGUAGE {\huge U}NDERSTANDING} {\huge S}YSTEMS}

		\vspace{0.5cm}

		{\Large \bfseries {FIRST PROJECT}}

		\vspace{1.5cm}
		
		{\large \textsc{Roberto Zen}}

		\vspace{1.0cm}
		
		{\today}

		\vspace{6.0cm}

		\small{The conducted work described in this report is currently released under the MIT license and it is available on \url{https://github.com/robzenn92/LUS}.}

		\vfill

	\end{center}

\end{titlepage}


\tableofcontents
\newpage

\section{Outline}

The report is structured as follows. Section describes ...
Section 2 describes how I used the \textit{FST} and \textit{GRM} tools for training and testing sequence labeling and it shows the results of applying these tools. Section 3 describes the same as section 2 but using the \textit{CRF++} tool.
Section 4 shows how text classification is made using \textit{Naive Bayes}.
The last section states the results and the conclusion of the conducted work.

\section{Data Analysis}

The given data set is composed as follows. Table 1 shows more details about the given files.

\begin{table}[h!]
	\vspace{0.2cm}
	\renewcommand*\arraystretch{1.5}
	\begin{center}
	\begin{tabular}{|l|l|l|l|}
		\hline
		File name & Used for & Word count & Token count \\ \hline

		NLSPARQL.test.feats.txt	& FST & 0 & 0 \\ \hline
		NLSPARQL.train.feats.txt & FST & 0 & 0 \\ \hline
		
		NLSPARQL.test.data	& CFF++ & 0 & 0 \\ \hline
		NLSPARQL.train.data	& CFF++ & 0 & 0 \\ \hline
		
		NLSPARQL.train.tok	& Naive Bayes & 0 & 0 \\ \hline
		NLSPARQL.train.utt.labels.txt & Naive Bayes & 0 & 0 \\ \hline
		NLSPARQL.test.tok	& Naive Bayes & 0 & 0 \\ \hline
		NLSPARQL.test.utt.labels.txt & Naive Bayes & 0 & 0 \\ \hline
		
	\end{tabular}
	\vspace{0.2cm}
	\caption{Details of the dataset.}
	\label{table:dataset_details}
	\end{center}
\end{table}



\section{Evaluation}

\section{Sequence Labeling in CRF++}

\section{Feature sets}

\section{Tool configuration}

\section{Conclusion}

% \bibliographystyle{unsrt}
% \bibliography{bibliography}

\end{document}